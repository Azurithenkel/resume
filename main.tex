%%%%%%%%%%%%%%%%%%%%%%%%%%%%%%%%%%%%%%%%
% Medium Length Professional CV
% LaTeX Template
% Version 2.0 (8/5/13)
%
% This template has been downloaded from:
% http://www.LaTeXTemplates.com
%
% Original author:
% Trey Hunner (http://www.treyhunner.com/)
%
% Important note:
% This template requires the resume.cls file to be in the same directory as the
% .tex file. The resume.cls file provides the resume style used for structuring the
% document.
%
%%%%%%%%%%%%%%%%%%%%%%%%%%%%%%%%%%%%%%%%%

%----------------------------------------------------------------------------------------
%	PACKAGES AND OTHER DOCUMENT CONFIGURATIONS
%----------------------------------------------------------------------------------------

\documentclass{format/resume} % Use the custom resume.cls style

\usepackage[left=0.75in,top=0.6in,right=0.75in,bottom=0.6in]{geometry} % Document margins

\name{Tom\'{a}\v{s} Pila\v{r}} % Your name
\address{94 James Green Road \\ Coventry, CV4 9SN, United Kingdom} % Your address
%\address{123 Pleasant Lane \\ City, State 12345} % Your secondary addess (optional)
\address{+44 $\cdot$ 7531 $\cdot$ 784896  \\ t.pilar@warwick.ac.uk} % Your phone number and email

\begin{document}

%----------------------------------------------------------------------------------------
%	EDUCATION SECTION
%----------------------------------------------------------------------------------------

\begin{rSection}{Education}
{\bf University of Warwick, UK} \hfill {October 2010 - Present} \\
PhD in Physics\\
Focus on Particle Physics, Thesis on Charm Mixing and Dalitz Plots\\
The study included a concurrent course on Transferrable Doctoral Skills \\
Spent 14 months in CERN on a Long Term Attachment

{\bf University of Warwick, UK} \hfill {October 2006 - June 2010} \\
MMathPhys upper second class with honors (2.1)\\
Fourth year focus on Particle Physics with a project on Charm Mixing\\
General areas of study include: Mathematical Analysis, Linear Algegra,
Knot Theory, Astrophysics, Statistical methods, Scientific Programming

{\bf Bilingual English-Slovak Grammar School of Milan Hod\v{z}a}
\hfill {September 2001 - June 2006} \\
A-Level equivalent school leaving examination in Mathematics with
higher Mathematics (A), Physics(A), IT(A), Chemistry(A), English
Language and Literature(A) and Slovak Language and Literature(B)
\end{rSection}

%----------------------------------------------------------------------------------------
%	POSITIONS OF RESPONSIBILITY
%----------------------------------------------------------------------------------------

\begin{rSection}{Positions of Responsibility}

\begin{rSubsection}{GAUSS Simulation Group}{ October 2011 - Present
  }{Decay Files repository manager}{LHCb Collaboration, CERN}
\item Part of service work includes organizing and managing a large database of configuration files for Monte Carlo production.
\item Developed tools that automated the validation of new
  configuration files leading to significant reduction in
  complexity for new users.
\end{rSubsection}

\begin{rSubsection}{EvtGen Development Group}{ October 2010 - Present }{Development and technical support}{University of Warwick, UK}
\item Updated the master configuration file for EvtGen heavy flavour
  generator listing every decay mode of every particle handled.
\item Developed automation tools that helped reduce the length of the
  regular update from 6 months to about two of weeks.
\item Commissioned and administrated the communications server and the hosted webpage (evtgen.warwick.ac.uk).
\end{rSubsection}

\begin{rSubsection}{LHCb UK Student Group}{ June 2012 - December 2012 }{Convener}{LHCb Collaboration, CERN}
\item Convened and organized talks for an LHCb CERN based group providing introductory and advanced talks aimed at PhD Students doing an internship at CERN.
\item Gave a 30' introductory talk on the LHCb Detector.
\end{rSubsection}

\begin{rSubsection}{Mathematics for Physicists Course}{ October 2010 - June 2010
  }{Teaching Assistant}{University of Warwick}
\item Taught two support classes of Mathematics for Physicists for 1st
  year Physics students.
\item Prepared students for the most difficult exam of 1st year Physics.
\end{rSubsection}

\begin{rSubsection}{University of Warwick Computing Society}{ February 2007 - January 2009 }{Co-President (Dual Presidency)}{University of Warwick, UK}
\item Presided over active society with both academic and leisure
  interests along with excellent IT equipment that provides technical services for majority of other student societies at university.
\item The society counted $\sim$ 150 members, during my term the society acquired significant amount of leisure equipment for weekly social gatherings.
\item Greatly enhanced the library of software available for weekly gatherings (including porting non-native software to Linux).
\item Organized and ran LAN parties for up to 60 people six times a
  year.
\item Restructured the executive commitee to better run the society -
  introduced the dual presidency principle.
\end{rSubsection}

\begin{rSubsection}{Student Associate Scheme for Secondary Schools}{ October 2007 - March
    2008 }{Student Associate}{University of Warwick}
\item Took part in a course learning to teach Mathematics at Secondary
  Schools
\item Underwent a two-week paid internship at Tile Hill Wood School
  and Language College
\end{rSubsection}
\end{rSection}

%----------------------------------------------------------------------------------------
%       INTERESTS AND HOBBIES
%----------------------------------------------------------------------------------------

\begin{rSection}{Hobbies, Interests and other skills}
\begin{tabular}{ @{} >{\bfseries}l @{\hspace{6ex}} l }
Computer Languages & C/C++, Python \\
Web Development & Apache, CSS, PHP, Django, MySQL \\
OS competence & Linux, Windows \\
Office Tools & MS Office, LibreOffice, LaTeX \\
Data Analysis & ROOT, RooFit, CERN software \\
Spoken Languages & Slovak, English (Bilingual), French (Intermediate)
\end{tabular}

{\bf Hobbies}\\
CERN Board Games Society\\
University of Warwick Computing Society\\
University of Warwick Science Fiction and Fantasy Society\\
Bilingual English Slovak Grammar School Debating Club
\end{rSection}

%----------------------------------------------------------------------------------------
%        RESEARCH INTERESTS
%----------------------------------------------------------------------------------------

\begin{rSection}{Research Interests}
{\bf Charm physics} \hfill {\bf Monte Carlo production} \\
Neutral charm meson mixing and CP violation \hfill Decay model implementation \\
Dalitz analysis, Multivariate analysis \hfill EvtGen development
\end{rSection}

%----------------------------------------------------------------------------------------
%	CONFERENCES AND PUBLICATIONS
%----------------------------------------------------------------------------------------

\begin{rSection}{Conferences and Publications}

\begin{rSubsection}{Thesis}{To be finished in March 2014}{Charm mixing in $D^{0}\rightarrow K_{S}\pi\pi$ at LHCb}{University of Warwick}
\item Analysis is expected to result in a published paper.
\item We use and cross-check three different extraction methods for charm mixing parameters.
\item The analysis was one of the flagships for an advanced technique
  for correction of lifetime bias at LHCb.
\end{rSubsection}

%------------------------------------------------

\begin{rSubsection}{Large Hadron Collider Physics Conference}{ May 2013}{Charm mixing and CP violation at LHCb}{Barcelona, Spain}
\item Gave a 15' parallel session talk on the state of the Charm mixing and CP violation analyses at LHCb in May 2013.
\end{rSubsection}

%-----------------------------------------------

\begin{rSubsection}{Institute of Physics HEP Conference}{ April 2013}{Measuring charm mixing parameters using a model-independent \hfill Liverpool, UK \\ technique in $D^{0} \rightarrow K_{S}hh$}{}
\item Gave a 20' talk presenting a model independent way of extracting mixing parameters using the $D^{0}\rightarrow K_{S}hh$ channel using Dalitz plots.
\item Produced and shown a poster on the same topic.
\end{rSubsection}

\end{rSection}

%----------------------------------------------------------------------------------------

\end{document}
